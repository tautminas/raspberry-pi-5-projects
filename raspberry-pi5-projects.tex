\documentclass[a4paper,12pt]{article}
\usepackage[T1]{fontenc}
\usepackage[utf8]{inputenc}
\usepackage{hyperref}
\usepackage{listings}
\usepackage{xcolor}
\usepackage{tocloft}

\title{Raspberry Pi 5 Projects}
\date{}

\hypersetup{
    colorlinks=true,
    linkcolor=blue,
    urlcolor=red,
    filecolor=green,
    pdfborder={0 0 0},
    citecolor=blue
}

\hypersetup{linkcolor=black}


\begin{document}

\maketitle

\renewcommand{\cftsecleader}{\cftdotfill{\cftdotsep}}

\tableofcontents

\section*{About}
\addcontentsline{toc}{section}{\protect\numberline{}About}

This document provides descriptions, instructions, and commands for projects using a standard Raspberry Pi 5 kit. Each project is inspired by a YouTube video, but this document refines and expands upon the content to offer clear, structured guidance for those who wish to replicate the projects. The purpose of collecting this information in a single document is as follows:

\begin{itemize}
\item Many YouTube videos are irrelevant or poorly made. This document carefully selects only the most useful ones.
\item Some videos lack key instructions or commands. By providing them here, users can avoid the need to retype everything manually.
\item Some videos require additional explanations or expanded details. This document enhances those projects to make them more practical and informative.
\end{itemize}

In IT, a "project" typically refers to a time-bound endeavor with a unique outcome. While this definition does not fully align with how the term is used in this document, it remains the closest approximation to describe the work presented here.

\section*{Raspberry Pi 5 preparation}
\addcontentsline{toc}{section}{\protect\numberline{}Raspberry Pi 5 preparation}

Add SSD to the PC using an SSD reader to install an operating system. Use the Raspberry Pi Imager, which is a tool that simplifies the process of installing an operating system onto your Raspberry Pi’s storage device. It has three main options:
\begin{itemize}
\item \textbf{Device}: Select the Raspberry Pi device (in this case, Raspberry Pi 5).
\item \textbf{Operating System}: Choose the OS based on the project requirements. The specific OS used for each project will be specified in the \textbf{Projects Overview} section.
\item \textbf{Storage}: Select the storage device, either a microSD card or an SSD, which will be used to boot and store data for the Raspberry Pi.
\end{itemize}

Raspberry Pi Imager will write the OS to the selected storage and allow you to configure basic settings such as hostname, Wi-Fi, and SSH. Before clicking \textbf{Next}, press \textbf{Ctrl + Shift + X} to customize settings under \textbf{General}, \textbf{Services}, and \textbf{Options}.

\section{Hosting a Dark Web Site}

\subsection{Overview}
Video: \href{https://www.youtube.com/watch?v=bllS9tkCkaM}{\textbf{\color{blue}i put a DARK WEB website on a Raspberry Pi!!}} \\
Channel: \href{https://www.youtube.com/c/NetworkChuck}{\textbf{\color{blue}NetworkChuck}} \\
Operating System: Raspberry Pi OS (64-bit)

\subsection{Steps}
\begin{enumerate}
    \item Update package lists and install Tor:
    \begin{lstlisting}[language=bash, breaklines=true, breakatwhitespace=true, columns=fullflexible]
    $ sudo apt update
    $ sudo apt install tor
    \end{lstlisting}

   \item Edit the Tor configuration file:
    \begin{lstlisting}[language=bash, breaklines=true, breakatwhitespace=true, columns=fullflexible]
    $ sudo nano /etc/tor/torrc
    \end{lstlisting}
    Uncomment the following lines:
    \begin{lstlisting}[language=bash, breaklines=true, breakatwhitespace=true, columns=fullflexible]
    HiddenServiceDir /var/lib/tor/hidden_service/
    HiddenServicePort 80 127.0.0.1:80
    \end{lstlisting}

    \item Restart the Tor service:
    \begin{lstlisting}[language=bash, breaklines=true, breakatwhitespace=true, columns=fullflexible]
    $ sudo service tor stop
    $ sudo service tor start
    $ sudo service tor status
    \end{lstlisting}

    \item Retrieve your site address:
    \begin{lstlisting}[language=bash, breaklines=true, breakatwhitespace=true, columns=fullflexible]
    $ sudo cat /var/lib/tor/hidden_service/hostname
    \end{lstlisting}

    \item Install and start Nginx:
    \begin{lstlisting}[language=bash, breaklines=true, breakatwhitespace=true, columns=fullflexible]
    $ sudo apt install nginx
    $ sudo service nginx start
    $ sudo service nginx status
    \end{lstlisting}

\item Modify Nginx configuration:
\begin{lstlisting}[language=bash, breaklines=true, breakatwhitespace=true, columns=fullflexible]
$ sudo nano /etc/nginx/nginx.conf
\end{lstlisting}

Uncomment the following lines:
\begin{lstlisting}[language=bash, breaklines=true, breakatwhitespace=true, columns=fullflexible]
port_in_redirect off;
server_name_in_redirect off;
\end{lstlisting}

Add the following line below "port\_in\_redirect off;":
\begin{lstlisting}[language=bash, breaklines=true, breakatwhitespace=true, columns=fullflexible]
server_tokens off;
\end{lstlisting}

    \item Restart Nginx:
    \begin{lstlisting}[language=bash, breaklines=true, breakatwhitespace=true, columns=fullflexible]
    $ sudo service nginx restart
    \end{lstlisting}

    \item Create a simple web page:
    \begin{lstlisting}[language=bash, breaklines=true, breakatwhitespace=true, columns=fullflexible]
    $ cd /var/www/html
    $ sudo rm index.nginx-debian.html
    $ sudo nano index.html
    \end{lstlisting}
    Add your HTML content and save the file.

\item Restart Nginx to apply changes:
\begin{lstlisting}[language=bash, breaklines=true, breakatwhitespace=true, columns=fullflexible]
$ sudo service nginx restart
\end{lstlisting}
Now, your site should be accessible via the generated .onion address!
   
\end{enumerate}

\section{Mining Monero Cryptocurrency}

\subsection{Overview}
Video: \href{https://www.youtube.com/watch?v=hHtGN_JzoP8}{\textbf{\color{blue}How to Mine Monero on Raspberry Pi}} \\
Pool: \href{https://moneroocean.stream/}{\textbf{\color{blue}Monero Ocean}} \\
Operating System: Raspberry Pi OS Lite (64-bit)

\subsection{Steps}
\begin{enumerate}
    \item Update package lists and install dependencies:
    \begin{lstlisting}[language=bash, breaklines=true, breakatwhitespace=true, columns=fullflexible]
    $ sudo apt update
    $ sudo apt install git build-essential cmake libuv1-dev libssl-dev libhwloc-dev -y
    \end{lstlisting}

    \item Clone the XMRig repository and build the miner:
    \begin{lstlisting}[language=bash, breaklines=true, breakatwhitespace=true, columns=fullflexible]
    $ git clone https://github.com/xmrig/xmrig.git
    $ cd xmrig
    $ mkdir build
    $ cd build
    $ cmake ..
    $ make
    \end{lstlisting}

    \item Choose one of the following mining options:
    \begin{enumerate}
        \item Standard mining:
        \begin{lstlisting}[language=bash, breaklines=true, breakatwhitespace=true, columns=fullflexible]
        $ ./xmrig -o gulf.moneroocean.stream:10128 -u <YOUR_WALLET_ADDRESS> -p <WORKER_NAME>
        \end{lstlisting}
        
        \item Mining in a detached session using tmux:
        \begin{lstlisting}[language=bash, breaklines=true, breakatwhitespace=true, columns=fullflexible]
        $ tmux
        $ ./xmrig -o gulf.moneroocean.stream:10128 -u <YOUR_WALLET_ADDRESS> -p <WORKER_NAME>
        \end{lstlisting}
        Press \texttt{Ctrl+b}, then \texttt{d} to detach.
        To reattach the session:
        \begin{lstlisting}[language=bash, breaklines=true, breakatwhitespace=true, columns=fullflexible]
        $ tmux attach
        \end{lstlisting}
        
        \item Automating mining with a startup script:
        \begin{lstlisting}[language=bash, breaklines=true, breakatwhitespace=true, columns=fullflexible]
        $ nano start_mining.sh
        \end{lstlisting}
        Add the following lines:
        \begin{lstlisting}[language=bash, breaklines=true, breakatwhitespace=true, columns=fullflexible]
        #!/bin/bash
        cd <BUILD_DIRECTORY_OF_XMRIG>
        tmux new-session -d -s xmrig_session './xmrig -o gulf.moneroocean.stream:10128 -u <YOUR_WALLET_ADDRESS> -p <WORKER_NAME>'
        \end{lstlisting}
        Save and exit, then make the script executable:
        \begin{lstlisting}[language=bash, breaklines=true, breakatwhitespace=true, columns=fullflexible]
        $ chmod +x start_mining.sh
        \end{lstlisting}
        Run the script:
        \begin{lstlisting}[language=bash, breaklines=true, breakatwhitespace=true, columns=fullflexible]
        $ ./start_mining.sh
        \end{lstlisting}
	Now, Monero mining will run in the background, and you can check the session using \texttt{tmux attach}.
    \end{enumerate}
\end{enumerate}

\section{Retropie}

\subsection{Overview}
Lorem ipsum dolor sit amet, consectetur adipiscing elit. Suspendisse potenti. Integer feugiat nisl vel lacus efficitur, a tincidunt eros vulputate.

\subsection{Steps}
\begin{enumerate}
    \item Lorem ipsum dolor sit amet, consectetur adipiscing elit.
    \item Lorem ipsum dolor sit amet, consectetur adipiscing elit.
\end{enumerate}

\section{NAS}

\subsection{Overview}
Lorem ipsum dolor sit amet, consectetur adipiscing elit. Suspendisse potenti. Integer feugiat nisl vel lacus efficitur, a tincidunt eros vulputate.

\subsection{Steps}
\begin{enumerate}
    \item Lorem ipsum dolor sit amet, consectetur adipiscing elit.
    \item Lorem ipsum dolor sit amet, consectetur adipiscing elit.
\end{enumerate}

\section{LLM}

\subsection{Overview}
Lorem ipsum dolor sit amet, consectetur adipiscing elit. Suspendisse potenti. Integer feugiat nisl vel lacus efficitur, a tincidunt eros vulputate.

\subsection{Steps}
\begin{enumerate}
    \item Lorem ipsum dolor sit amet, consectetur adipiscing elit.
    \item Lorem ipsum dolor sit amet, consectetur adipiscing elit.
\end{enumerate}

\section{Pihole}

\subsection{Overview}
Lorem ipsum dolor sit amet, consectetur adipiscing elit. Suspendisse potenti. Integer feugiat nisl vel lacus efficitur, a tincidunt eros vulputate.

\subsection{Steps}
\begin{enumerate}
    \item Lorem ipsum dolor sit amet, consectetur adipiscing elit.
    \item Lorem ipsum dolor sit amet, consectetur adipiscing elit.
\end{enumerate}

\end{document}
