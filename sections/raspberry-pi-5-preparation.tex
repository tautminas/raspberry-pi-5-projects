\section*{\phantomsection Raspberry Pi 5 Preparation}
\addcontentsline{toc}{section}{Raspberry Pi 5 Preparation}

Before using your Raspberry Pi, you must first install an operating system onto a microSD card. Follow these steps to set it up:
\begin{enumerate}
\item Insert the microSD card into your PC (not the Raspberry Pi) using an SD adapter. Depending on the adapter, plug it into either the SD card slot or a USB port on your computer.
\item Open the \href{https://www.raspberrypi.com/software}{\textbf{\color{blue}Raspberry Pi Imager}} application. This tool simplifies the process of installing an operating system onto your Raspberry Pi’s storage. It has three main options:
\begin{itemize}
\item Device: Select your Raspberry Pi device (in this case, Raspberry Pi 5).
\item Operating System: Choose the OS based on the project’s requirements. The specific OS will be mentioned in the overview subsection of the project section.
\item Storage: Select the storage device (in this case, a microSD card) that will be used to boot the Raspberry Pi and store data. Be careful, as this process will erase all existing data on the selected storage device.
\end{itemize}
\item Customize the settings to enable remote access to the Raspberry Pi by pressing \texttt{Ctrl + Shift + X}. Then, set the hostname, username, password, configure wireless LAN, set locale settings and enable SSH.
\item Proceed with the OS installation by selecting \textbf{Next}. The Raspberry Pi Imager will then write the OS to the selected storage device.
\item Once the OS is successfully written to the microSD card, insert it into the Raspberry Pi and power it on. Your Raspberry Pi will now be ready to use.
\end{enumerate}

For most projects, you'll need to connect to the Raspberry Pi remotely. To do so, follow these steps:
\begin{enumerate}
\item Use a tool like \href{https://www.fing.com/}{\textbf{\color{blue}Fing}}, available for both mobile and desktop, to find the Raspberry Pi's IP address. Fing detects devices on your local network and displays their IP addresses. 
\item Once you have the IP address, connect to the Raspberry Pi via SSH. Use a tool like \href{https://www.chiark.greenend.org.uk/~sgtatham/putty/}{\textbf{\color{blue}PuTTY}}, a free SSH client for Windows. Launch PuTTY, enter the Raspberry Pi's IP address, select SSH as the connection type and click \textbf{Open}. Log in using your customized credentials (username and password) if you have set them, or use the default credentials (username: pi, password: raspberry) otherwise.
\end{enumerate}

After logging into your Raspberry Pi 5, either remotely or by directly accessing it, you need to run two essential commands that are recommended for every project:
\begin{enumerate}
\item \texttt{sudo apt update}: This command updates the list of available packages and their versions from the repositories.
\item \texttt{sudo apt upgrade}: This command upgrades all the installed packages to their latest available versions.
\end{enumerate}

You can run both commands at once by using:
\begin{verbatim} sudo apt update && sudo apt upgrade -y \end{verbatim}

The \texttt{-y} flag automatically confirms all prompts during the upgrade process, so you won't need to manually approve each step. This may take some time depending on your internet speed and the number of packages that need to be upgraded.
