\section{Hosting a Dark Web Site}

\textbf{Video}: \href{https://www.youtube.com/watch?v=bllS9tkCkaM}{\textbf{\color{blue}i put a DARK WEB website on a Raspberry Pi!!}} (Channel: \href{https://www.youtube.com/@NetworkChuck}{\textbf{\color{blue}NetworkChuck}})

\vspace{0.5cm}

\noindent \textbf{Operating System}: Raspberry Pi OS Lite (64-bit), remote connection.

\vspace{0.5cm}

\noindent \textbf{Steps:}

\begin{enumerate}
\item Update package lists and install Tor:
\begin{lstlisting}[language=bash, breaklines=true, breakatwhitespace=true, columns=fullflexible]
$ sudo apt update
$ sudo apt install tor
\end{lstlisting}

\item Edit the Tor configuration file:
\begin{lstlisting}[language=bash, breaklines=true, breakatwhitespace=true, columns=fullflexible]
$ sudo nano /etc/tor/torrc
\end{lstlisting}
Uncomment the following lines:
\begin{lstlisting}[language=bash, breaklines=true, breakatwhitespace=true, columns=fullflexible]
HiddenServiceDir /var/lib/tor/hidden_service/
HiddenServicePort 80 127.0.0.1:80
\end{lstlisting}

\item Restart the Tor service:
\begin{lstlisting}[language=bash, breaklines=true, breakatwhitespace=true, columns=fullflexible]
$ sudo service tor stop
$ sudo service tor start
$ sudo service tor status
\end{lstlisting}

\item Obtain your site address:
\begin{lstlisting}[language=bash, breaklines=true, breakatwhitespace=true, columns=fullflexible]
$ sudo cat /var/lib/tor/hidden_service/hostname
\end{lstlisting}

\item Install and start Nginx:
\begin{lstlisting}[language=bash, breaklines=true, breakatwhitespace=true, columns=fullflexible]
$ sudo apt install nginx
$ sudo service nginx start
$ sudo service nginx status
\end{lstlisting}
You can now check your website using the .onion address from the previous step.

\item Modify Nginx configuration:
\begin{lstlisting}[language=bash, breaklines=true, breakatwhitespace=true, columns=fullflexible]
$ sudo nano /etc/nginx/nginx.conf
\end{lstlisting}
Uncomment the following lines:
\begin{lstlisting}[language=bash, breaklines=true, breakatwhitespace=true, columns=fullflexible]
port_in_redirect off;
server_name_in_redirect off;
\end{lstlisting}

Add the following line below "port\_in\_redirect off;":
\begin{lstlisting}[language=bash, breaklines=true, breakatwhitespace=true, columns=fullflexible]
server_tokens off;
\end{lstlisting}

\item Restart Nginx to apply changes:
\begin{lstlisting}[language=bash, breaklines=true, breakatwhitespace=true, columns=fullflexible]
$ sudo service nginx restart
\end{lstlisting}
Visit the site using your .onion address to check if the site works after the configuration changes.

\item Create a simple web page:
\begin{lstlisting}[language=bash, breaklines=true, breakatwhitespace=true, columns=fullflexible]
$ cd /var/www/html
$ sudo rm index.nginx-debian.html
$ sudo nano index.html
\end{lstlisting}
Add your HTML content and save the file.

\item Restart Nginx to apply changes:
\begin{lstlisting}[language=bash, breaklines=true, breakatwhitespace=true, columns=fullflexible]
$ sudo service nginx restart
\end{lstlisting}
Your site should now be accessible via your .onion address!
\end{enumerate}
