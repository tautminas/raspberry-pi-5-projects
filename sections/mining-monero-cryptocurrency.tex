\section{Mining Monero Cryptocurrency}

\noindent \textbf{Video}: \href{https://www.youtube.com/watch?v=hHtGN_JzoP8}{\textbf{\color{blue}How to Mine Monero on Raspberry Pi}} (Channel: \href{https://www.youtube.com/@NetworkChuck}{\textbf{\color{blue}NetworkChuck}})

\vspace{0.5cm}

\noindent \textbf{Operating System}: Raspberry Pi OS Lite (64-bit), remote connection.

\vspace{0.5cm}

\noindent \textbf{Steps:}
\begin{enumerate}
\item Update package lists and install dependencies:
\begin{lstlisting}[language=bash, breaklines=true, breakatwhitespace=true, columns=fullflexible]
$ sudo apt update
$ sudo apt install git build-essential cmake libuv1-dev libssl-dev libhwloc-dev -y
\end{lstlisting}

\item Clone the XMRig repository and build the miner:
\begin{lstlisting}[language=bash, breaklines=true, breakatwhitespace=true, columns=fullflexible]
$ git clone https://github.com/xmrig/xmrig.git
$ cd xmrig
$ mkdir build
$ cd build
$ cmake ..
$ make
\end{lstlisting}

\item Create a Monero (XMR) wallet: Download and install the Monero GUI Wallet from \href{https://www.getmonero.org}{\textbf{\color{blue}getmonero.org}}, choose Simple Mode, create a new wallet, name it, back up the mnemonic seed, set a password, and find your wallet address under the Account tab.

\item Start mining Monero:  
\begin{lstlisting}[language=bash, breaklines=true, breakatwhitespace=true, columns=fullflexible]
$ ./xmrig -o gulf.moneroocean.stream:10128 -u <YOUR_WALLET_ADDRESS> -p <WORKER_NAME>
\end{lstlisting}

Use the following keys to monitor mining in real time:  
\begin{itemize}
\item \texttt{H} — Show hashrate  
\item \texttt{S} — Display the number of accepted shares  
\item \texttt{C} — Check connection status  
\end{itemize}

To track your mining performance, visit \href{https://moneroocean.stream/}{\textbf{\color{blue}Monero Ocean}}, enter your wallet address and view your statistics.

\end{enumerate}

\vspace{0.5cm}

\noindent \textbf{Optional Enhancements:}
\begin{itemize}
\item Running the miner in a detached session with tmux:
\begin{lstlisting}[language=bash, breaklines=true, breakatwhitespace=true, columns=fullflexible]
$ tmux
$ ./xmrig -o gulf.moneroocean.stream:10128 -u <YOUR_WALLET_ADDRESS> -p <WORKER_NAME>
\end{lstlisting}
Press \texttt{Ctrl+b}, then \texttt{d} to detach the session while keeping it running in the background.

To reattach the session use the following command:
\begin{lstlisting}[language=bash, breaklines=true, breakatwhitespace=true, columns=fullflexible]
$ tmux attach
\end{lstlisting}
        
\item Automating mining with a startup script:
\begin{lstlisting}[language=bash, breaklines=true, breakatwhitespace=true, columns=fullflexible]
$ nano start_mining.sh
\end{lstlisting}
Add the following lines:
\begin{lstlisting}[language=bash, breaklines=true, breakatwhitespace=true, columns=fullflexible]
#!/bin/bash
cd <BUILD_DIRECTORY_OF_XMRIG>
tmux new-session -d -s xmrig_session './xmrig -o gulf.moneroocean.stream:10128 -u <YOUR_WALLET_ADDRESS> -p <WORKER_NAME>'
\end{lstlisting}
Save and exit, then make the script executable:
\begin{lstlisting}[language=bash, breaklines=true, breakatwhitespace=true, columns=fullflexible]
$ chmod +x start_mining.sh
\end{lstlisting}
Run the script:
\begin{lstlisting}[language=bash, breaklines=true, breakatwhitespace=true, columns=fullflexible]
$ ./start_mining.sh
\end{lstlisting}
The script ensures that Monero mining starts in a detached tmux session, allowing it to run persistently in the background. The session can be reattached using \texttt{tmux attach}.
\end{itemize}
