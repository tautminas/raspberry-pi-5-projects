\section*{\phantomsection About}
\addcontentsline{toc}{section}{About}

This document provides descriptions, instructions and commands for projects using a Raspberry Pi 5. Each project is inspired by a YouTube video, but this document refines and expands upon the content to offer clear guidance for those who wish to replicate the projects. The purpose of collecting this information in a single document is as follows:

\begin{itemize}
\item Some videos lack quality or are no longer relevant. This document selects the useful ones.
\item Some videos do not provide commands in written form. By providing them here, readers can avoid needing to manually retype everything.
\item Some videos could benefit from additional suggestions or improvements. This document expands on those projects to make them more practical.
\end{itemize}

The projects in this document were created using the Raspberry Pi 5 Desktop Kit. However, in most cases, a Raspberry Pi 5 Starter Kit provides all the necessary components to follow along. Additionally, these projects are not specific to the Raspberry Pi 5 as most of them can be completed using other Raspberry Pi models as well.  

Some instructions require specific values, for example, an IP address or cryptocurrency wallet address. These values are represented in the format \texttt{\textless EXAMPLE\_VALUE\textgreater}, with angle brackets and capitalization to indicate that they must be replaced with actual values. Blindly copying and pasting commands without replacing these placeholders will result in errors. Pay attention to the instructions and substitute the placeholders with the appropriate values.

In the realm of project management, a project is defined as a time-bound endeavour with a unique outcome. While this definition does not fully align with how it is used in the document, it remains the closest approximation to describe the work presented here.