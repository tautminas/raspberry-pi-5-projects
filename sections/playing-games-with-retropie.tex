\section{Playing Games with RetroPie}

\noindent \textbf{Video}: \href{https://www.youtube.com/watch?v=AaseHnf0k2o}{\textbf{\color{blue}RetroPie: A Raspberry Pi Gaming Machine}} (Channel: \href{https://www.youtube.com/@NetworkChuck}{\textbf{\color{blue}NetworkChuck}})

\vspace{0.5cm}

\noindent \textbf{Operating System}: Raspberry Pi OS Lite (64-bit)

\vspace{0.5cm}

\noindent \textbf{Steps:}

\begin{enumerate}
\item Prepare the system and install essential packages for RetroPie setup:
\begin{lstlisting}[language=bash, breaklines=true, breakatwhitespace=true, columns=fullflexible]
$ sudo apt update && sudo apt upgrade -y
$ sudo apt install git lsb-release
\end{lstlisting}

\item Clone the RetroPie setup script:
\begin{lstlisting}[language=bash, breaklines=true, breakatwhitespace=true, columns=fullflexible]
$ cd
$ git clone https://github.com/retropie/retropie-setup.git
$ cd retropie-setup
$ chmod +x retropie_setup.sh
\end{lstlisting}

\item Run the setup script:
\begin{lstlisting}[language=bash, breaklines=true, breakatwhitespace=true, columns=fullflexible]
$ sudo ./retropie_setup.sh
\end{lstlisting}
Follow the prompts: OK → OK → Basic Install → Yes. The installation will take a while.

Once complete, go to Configuration / tools → autostart → Start EmulationStation at boot → OK.

Then, go to bashwelcometweak → Install Bash Welcome Tweak → OK → Exit.

\item Reboot the system:
\begin{lstlisting}[language=bash, breaklines=true, breakatwhitespace=true, columns=fullflexible]
$ sudo reboot
\end{lstlisting}

\item Setup the keyboard:
\begin{table}[H]
\centering
\begin{tabular}{|l|l|}
\hline
\textbf{Gamepad Input} & \textbf{Keyboard Key} \\
\hline
D-PAD UP & Arrow Up \\
D-PAD DOWN & Arrow Down \\
D-PAD LEFT & Arrow Left \\
D-PAD RIGHT & Arrow Right \\
\hline
START & Enter \\
SELECT & Tab \\
BUTTON A / EAST & Spacebar \\
BUTTON B / SOUTH & Left Shift \\
BUTTON X / NORTH & Z \\
BUTTON Y / WEST & X \\
\hline
LEFT SHOULDER & Q \\
RIGHT SHOULDER & E \\
LEFT TRIGGER & R \\
RIGHT TRIGGER & T \\
LEFT THUMB & C \\
RIGHT THUMB & V \\
\hline
LEFT ANALOG UP & W \\
LEFT ANALOG DOWN & S \\
LEFT ANALOG LEFT & A \\
LEFT ANALOG RIGHT & D \\
RIGHT ANALOG UP & I \\
RIGHT ANALOG DOWN & K \\
RIGHT ANALOG LEFT & J \\
RIGHT ANALOG RIGHT & L \\
\hline
HOTKEY ENABLE & Left Control (Ctrl) \\
\hline
\end{tabular}
\caption{Suggested keyboard mapping for RetroPie}
\end{table}

\item Configure Raspberry Pi settings in RetroPie menu in the RASPI-CONFIG option:
\begin{itemize}
\item System Options → Hostname → Enter a hostname
\item System Options → Password → Enter a new password → OK
\item Localisation Options → Timezone → Select location → Select timezone
\item Localisation Options → WLAN Country → Select country
\item Interface Options → SSH → Yes
\end{itemize}

\item Select the WIFI option in the RetroPie menu → Connect to WiFi network → Pick network and enter the password

\item Download ROMs for your preferred games using another device. ROMs can be downloaded from \href{https://www.emulatorgames.net}{\textbf{\color{blue}emulatorgames.net}} and \href{https://www.romsgames.net}{\textbf{\color{blue}romsgames.net}}. Check the list of \href{https://retropie.org.uk/docs/Supported-Systems/}{\textbf{\color{blue}RetroPie Supported Systems}} to ensure game compatibility before downloading.

\item In the RetroPie menu select SHOW IP. Transfer ROMs to the Raspberry Pi using SCP from the device where the ROMs are stored.
\begin{lstlisting}[language=bash, breaklines=true, breakatwhitespace=true, columns=fullflexible]
$ scp -r <PATH_TO_ROMS_DIRECTORY>/* <USERNAME>@<RASPBERRY_PI_IP>:~/RetroPie/roms/
\end{lstlisting}

\item Organize the ROMs into their appropriate emulator folders on the Raspberry Pi.
Pressing \textbf{F4} will take you to the command line interface (CLI). For example, PlayStation games should go into \texttt{~/RetroPie/roms/psx}. Use \texttt{cd} command to move ROMs to the needed location. To return to EmulationStation use the \texttt{emulationstation} command.

\item Personalize game lists in the MAIN MENU:
\begin{itemize}
\item UI SETTINGS → GAME LIST VIEW STYLE → GRID → set IGNORE ARTICLES (NAME SORT ONLY) to OFF
\item SCRAPER → SCRAPE FROM → SCREENSCRAPER → SCRAPE NOW → Set SYSTEMS to all → set USER DECIDES ON CONFLICTS to off → Start
\end{itemize}

\item Now you can play the games. If desired, use these hotkeys while playing:
\begin{itemize}
\item \textbf{HOTKEY + Start}: Exit game
\item \textbf{HOTKEY + Right Shoulder}: Save game
\item \textbf{HOTKEY + Left Shoulder}: Load game
\item \textbf{HOTKEY + D-PAD Right}: Change save/load slot
\item \textbf{HOTKEY + Button B}: Reset game
\item \textbf{HOTKEY + Button X}: Open RetroArch menu
\end{itemize}

\end{enumerate}

\vspace{0.5cm}

\noindent \textbf{Optional Enhancements:}
\begin{itemize}

\item Enable cheats: Open RetroArch menu (HOTKEY + Button X) → Settings → User Interface → set Show Advanced Settings to ON → Go back to User Interface → Menu Item Visibility → set Show 'Online Updater' to ON → Go back to Main menu of RetroArch → Online Updater → Update Cheats → Go back to the Main menu of RetroArch → Quick Menu → Cheats → Load Cheat File (Replace) → Select console → Turn ON the desired cheats by selecting each one and setting Enable to ON → Apply Changes.

\item Enable achievements:
\begin{enumerate}
\item Sign up at \href{https://retroachievements.org}{\textbf{\color{blue}RetroAchievements}}.
\item Open the game on the Raspberry Pi in which you would like to turn on the achievements.
\item Open RetroArch menu (HOTKEY + Button X) → Settings → Achievements → Turn it ON → Select Username and put in value → Select Password and put in value → set Unlock Sound to ON.
\item Go back to the Main menu of RetroArch → Configuration File → Save Current Configuration → Quit RetroArch.
\end{enumerate}

\end{itemize}