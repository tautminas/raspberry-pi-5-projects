\section{Playing Games with RetroPie}

\subsection{Overview}
Video: \href{https://www.youtube.com/watch?v=AaseHnf0k2o}{\textbf{\color{blue}RetroPie: A Raspberry Pi Gaming Machine}} \\
Channel: \href{https://www.youtube.com/@NetworkChuck}{\textbf{\color{blue}NetworkChuck}} \\
Operating System: Raspberry Pi OS Lite (64-bit)

\subsection{Steps}

\begin{enumerate}
    \item Update and upgrade packages:
    \begin{lstlisting}[language=bash, breaklines=true, breakatwhitespace=true, columns=fullflexible]
    $ sudo apt update
    $ sudo apt upgrade -y
    $ sudo apt install git lsb-release
    \end{lstlisting}

   \item Clone the RetroPie setup script:
    \begin{lstlisting}[language=bash, breaklines=true, breakatwhitespace=true, columns=fullflexible]
    $ cd
    $ git clone https://github.com/retropie/retropie-setup.git
    $ cd retropie-setup
    $ chmod +x retropie_setup.sh
    \end{lstlisting}

    \item Run the setup script:
    \begin{lstlisting}[language=bash, breaklines=true, breakatwhitespace=true, columns=fullflexible]
    $ sudo ./retropie_setup.sh
    \end{lstlisting}
    Navigate through the setup: OK → OK → Basic install → Yes. The installation process will take a while. After completion, navigate to Configuration/Tools → Autostart → OK → bashwelcometweak → OK → Exit.

    \item Reboot the system:
    \begin{lstlisting}[language=bash, breaklines=true, breakatwhitespace=true, columns=fullflexible]
    $ sudo reboot
    \end{lstlisting}

    \item Transfer ROMs to RetroPie:
    ROMs can be downloaded from \href{https://www.emulatorgames.net}{\textbf{\color{blue}emulatorgames.net}} and \href{https://www.romsgames.net}{\textbf{\color{blue}romsgames.net}}. Use SCP to transfer ROMs from your desktop:
    \begin{lstlisting}[language=bash, breaklines=true, breakatwhitespace=true, columns=fullflexible]
    $ scp -r ~/Desktop/roms/* username@<YOUR_IP>:~/RetroPie/roms/
    \end{lstlisting}

    \item Configure Raspberry Pi settings in RASPI-CONFIG:
    \begin{itemize}
        \item System Options → Set Hostname, Change Password
        \item Localisation Options → Set Timezone, WLAN Country
        \item Interface Options → Enable SSH
    \end{itemize}

    \item Connect to WiFi (outside RASPI-CONFIG) and restart.

    \item Entering and exiting EmulationStation:
    Pressing \textbf{F4} will take you to the command line interface (CLI). To return to EmulationStation, enter:
    \begin{lstlisting}[language=bash, breaklines=true, breakatwhitespace=true, columns=fullflexible]
    $ emulationstation
    \end{lstlisting}
    ROMs must be placed in their respective emulator directories, e.g., \texttt{~/RetroPie/roms/psx} for PlayStation games.
    
    \item Reference supported systems at \href{https://retropie.org.uk/docs/Supported-Systems/}{\textbf{\color{blue}RetroPie Supported Systems}}.
    
    \item Configure input mappings for controllers.
    
    \item Set up hotkeys:
    \begin{itemize}
        \item \textbf{HOTKEY + Start}: Exit game
        \item \textbf{HOTKEY + Right Shoulder}: Save game
        \item \textbf{HOTKEY + Left Shoulder}: Load game
        \item \textbf{HOTKEY + D-PAD Right}: Change save/load slot
        \item \textbf{HOTKEY + Button B}: Reset game
        \item \textbf{HOTKEY + Button X}: Open RetroArch menu
    \end{itemize}
    
    \item Personalize game lists:
    \begin{itemize}
        \item UI Settings → Set Game List View Style to Grid
        \item Scraper → Scrape from ScreenScraper → Scrape Now
    \end{itemize}
    
    \item Enable cheats:
    \begin{enumerate}
        \item Open RetroArch menu (HOTKEY + Button X)
        \item Enable Advanced Settings
        \item Update Cheats via Online Updater
        \item Load Cheat File and apply changes
    \end{enumerate}

    \item Enable achievements:
    \begin{enumerate}
        \item Register at \href{https://retroachievements.org}{\textbf{\color{blue}RetroAchievements}}
        \item Configure username and password in RetroArch Settings
        \item Save configuration and restart
    \end{enumerate}
\end{enumerate}