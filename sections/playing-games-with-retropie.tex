\section{Playing Games with RetroPie}

\noindent \textbf{Video}: \href{https://www.youtube.com/watch?v=AaseHnf0k2o}{\textbf{\color{blue}RetroPie: A Raspberry Pi Gaming Machine}} (Channel: \href{https://www.youtube.com/@NetworkChuck}{\textbf{\color{blue}NetworkChuck}})

\vspace{0.5cm}

\noindent \textbf{Operating System}: Raspberry Pi OS Lite (64-bit)

\vspace{0.5cm}

\noindent \textbf{Steps:}

\begin{enumerate}
\item Prepare the system and install essential packages for RetroPie setup:  
\begin{lstlisting}[language=bash, breaklines=true, breakatwhitespace=true, columns=fullflexible]
$ sudo apt update && sudo apt upgrade -y  
$ sudo apt install git lsb-release
\end{lstlisting}

\item Clone the RetroPie setup script:
\begin{lstlisting}[language=bash, breaklines=true, breakatwhitespace=true, columns=fullflexible]
$ cd
$ git clone https://github.com/retropie/retropie-setup.git
$ cd retropie-setup
$ chmod +x retropie_setup.sh
\end{lstlisting}

\item Run the setup script:
\begin{lstlisting}[language=bash, breaklines=true, breakatwhitespace=true, columns=fullflexible]
$ sudo ./retropie_setup.sh
\end{lstlisting}
Navigate through the setup: OK → OK → Basic install → Yes. The installation process will take a while. After completion, navigate to Configuration/Tools → Autostart → OK → bashwelcometweak → OK → Exit.

\item Reboot the system:
\begin{lstlisting}[language=bash, breaklines=true, breakatwhitespace=true, columns=fullflexible]
$ sudo reboot
\end{lstlisting}

\item Setup the keyboard:  
\begin{itemize}
    \item \textbf{D-PAD UP} → Arrow Up  
    \item \textbf{D-PAD DOWN} → Arrow Down  
    \item \textbf{D-PAD LEFT} → Arrow Left  
    \item \textbf{D-PAD RIGHT} → Arrow Right  
    \item \textbf{START} → Enter  
    \item \textbf{SELECT} → Tab  
    \item \textbf{BUTTON A / EAST} → Spacebar  
    \item \textbf{BUTTON B / SOUTH} → Left Shift  
    \item \textbf{BUTTON X / NORTH} → Z  
    \item \textbf{BUTTON Y / WEST} → X  
    \item \textbf{LEFT SHOULDER} → Q  
    \item \textbf{RIGHT SHOULDER} → E  
    \item \textbf{LEFT TRIGGER} → R  
    \item \textbf{RIGHT TRIGGER} → T  
    \item \textbf{LEFT THUMB} → C  
    \item \textbf{RIGHT THUMB} → V  
    \item \textbf{LEFT ANALOG UP} → W  
    \item \textbf{LEFT ANALOG DOWN} → S  
    \item \textbf{LEFT ANALOG LEFT} → A  
    \item \textbf{LEFT ANALOG RIGHT} → D  
    \item \textbf{RIGHT ANALOG UP} → I  
    \item \textbf{RIGHT ANALOG DOWN} → K  
    \item \textbf{RIGHT ANALOG LEFT} → J  
    \item \textbf{RIGHT ANALOG RIGHT} → L  
    \item \textbf{HOTKEY ENABLE} → Left Control (Ctrl)  
\end{itemize}

\item Configure Raspberry Pi settings in RASPI-CONFIG:
\begin{itemize}
\item System Options → Set Hostname, Change Password
\item Localisation Options → Set Timezone, WLAN Country
\item Interface Options → Enable SSH
\end{itemize}

\item Connect to WiFi (outside RASPI-CONFIG) and restart.

\item Download ROMs for prefered games on a different device. ROMs can be downloaded from \href{https://www.emulatorgames.net}{\textbf{\color{blue}emulatorgames.net}} and \href{https://www.romsgames.net}{\textbf{\color{blue}romsgames.net}}. Reference supported systems at \href{https://retropie.org.uk/docs/Supported-Systems/}{\textbf{\color{blue}RetroPie Supported Systems}}.

\item Transfer ROMs to RetroPie using SCP from the device with ROMs:
\begin{lstlisting}[language=bash, breaklines=true, breakatwhitespace=true, columns=fullflexible]
$ scp -r ~/Desktop/roms/* <USERNAME>@<RASPBERRY_PI_IP>:~/RetroPie/roms/
\end{lstlisting}
    
\item Move ROMs on the Raspberry Pi to their needed directories
Pressing \textbf{F4} will take you to the command line interface (CLI). To return to EmulationStation, enter:
\begin{lstlisting}[language=bash, breaklines=true, breakatwhitespace=true, columns=fullflexible]
$ emulationstation
\end{lstlisting}
ROMs must be placed in their respective emulator directories, e.g., \texttt{~/RetroPie/roms/psx} for PlayStation games. Use cd command to move ROMs to the needed location.

\item Personalize game lists in main menu:
\begin{itemize}
\item UI Settings → Game List View Style → Grid
\item Scraper → Scrape from → ScreenScraper → Scrape Now → Set systems to all and user decides on conflicts off → Start
\end{itemize}
 
\noindent \textbf{Optional Enhancements:}
\begin{itemize}

\item Use hotkeys while playing:
\begin{itemize}
\item \textbf{HOTKEY + Start}: Exit game
\item \textbf{HOTKEY + Right Shoulder}: Save game
\item \textbf{HOTKEY + Left Shoulder}: Load game
\item \textbf{HOTKEY + D-PAD Right}: Change save/load slot
\item \textbf{HOTKEY + Button B}: Reset game
\item \textbf{HOTKEY + Button X}: Open RetroArch menu
\end{itemize}

\item Enable cheats:
\begin{enumerate}
\item Open RetroArch menu (HOTKEY + Button X)
\item Enable Advanced Settings
\item Update Cheats via Online Updater
\item Load Cheat File and apply changes
\end{enumerate}

\item Enable achievements:
\begin{enumerate}
\item Register at \href{https://retroachievements.org}{\textbf{\color{blue}RetroAchievements}}
\item Configure username and password in RetroArch Settings
\item Save configuration and restart
\end{enumerate}

\end{itemize}

\end{enumerate}


