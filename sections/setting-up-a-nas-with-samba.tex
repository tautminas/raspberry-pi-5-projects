\section{Setting Up a NAS with Samba}

\noindent \textbf{Video}: \href{https://www.youtube.com/watch?v=8QxJWW0mjAs}{\textbf{\color{blue}Pi Network File Share to Windows \& More | Pi NAS/SMB | Raspberry Pi Guide}} (Channel: \href{https://www.youtube.com/@TroubleChute}{\textbf{\color{blue}TroubleChute}})

\vspace{0.5cm}

\noindent \textbf{Operating System}: Raspberry Pi OS Lite (64-bit), remote connection.

\vspace{0.5cm}

\noindent \textbf{Steps:}

\begin{enumerate}
\item Update your system and install Samba:
\begin{lstlisting}[language=bash, breaklines=true, breakatwhitespace=true, columns=fullflexible]
$ sudo apt update && sudo apt upgrade -y
$ sudo apt install samba samba-common-bin -y
\end{lstlisting}

\item Create a shared directory:
\begin{lstlisting}[language=bash, breaklines=true, breakatwhitespace=true, columns=fullflexible]
$ mkdir ~/shared
\end{lstlisting}

\item Edit the Samba configuration file:
\begin{lstlisting}[language=bash, breaklines=true, breakatwhitespace=true, columns=fullflexible]
$ sudo nano /etc/samba/smb.conf
\end{lstlisting}
Add the following at the end of the file:
\begin{lstlisting}[language=bash, breaklines=true, breakatwhitespace=true, columns=fullflexible]
[shared]
path=/home/<USERNAME>/shared
writable=Yes
create mask=0666
directory mask=0666
public=no
\end{lstlisting}

\item Set a Samba password for your user:
\begin{lstlisting}[language=bash, breaklines=true, breakatwhitespace=true, columns=fullflexible]
$ sudo smbpasswd -a <USERNAME>
\end{lstlisting}

\item Restart the Samba service:
\begin{lstlisting}[language=bash, breaklines=true, breakatwhitespace=true, columns=fullflexible]
$ sudo systemctl restart smbd
\end{lstlisting}

\item Find your Raspberry Pi's IP address:
\begin{lstlisting}[language=bash, breaklines=true, breakatwhitespace=true, columns=fullflexible]
$ hostname -I
\end{lstlisting}

\item To access the shared folder from Windows, follow these steps:
\begin{enumerate}
\item Open \textbf{This PC}.
\item Right-click on an empty area and select \textbf{Add a network location}.
\item Proceed through the setup by clicking \textbf{Next} at each step until you reach the \textbf{Specify the location of your website} screen.
\item Enter the network path in the format: \texttt{\\\textbackslash\textbackslash<RASPBERRY\_PI\_IP>\textbackslash shared}, then click \textbf{Next}.
\end{enumerate}


\end{enumerate}