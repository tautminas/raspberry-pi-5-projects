\section{Setting Up Pi-hole for Ad Blocking}

\noindent \textbf{Video}: \href{https://www.youtube.com/watch?v=cE21YjuaB6o}{\textbf{\color{blue}World's Greatest Pi-hole Tutorial - Easy Raspberry Pi Project!}} (Channel: \href{https://www.youtube.com/@CrosstalkSolutions}{\textbf{\color{blue}Crosstalk Solutions}})

\vspace{0.5cm}

\noindent \textbf{Operating System}: Raspberry Pi OS Lite (64-bit), remote connection.

\vspace{0.5cm}

\noindent \textbf{Steps:}

\begin{enumerate}
\item Prepare the system and install the DHCP client daemon (needed for a static IP setup):
\begin{lstlisting}[language=bash, breaklines=true, breakatwhitespace=true, columns=fullflexible]
$ sudo apt update && sudo apt upgrade -y
$ sudo apt install dhcpcd
\end{lstlisting}
   
\item Configure a static IP address:
\begin{lstlisting}[language=bash, breaklines=true, breakatwhitespace=true, columns=fullflexible]
$ sudo nano /etc/dhcpcd.conf
\end{lstlisting}

Add the following lines at the end of the file:
\begin{lstlisting}[language=bash, breaklines=true, breakatwhitespace=true, columns=fullflexible]
interface wlan0
static ip_address=<YOUR_IP>/24
static routers=<YOUR_ROUTER_IP>
static domain_name_servers=<DNS_SERVER_1> <DNS_SERVER_2>
\end{lstlisting}

Save the file and reboot the Raspberry Pi:
\begin{lstlisting}[language=bash, breaklines=true, breakatwhitespace=true, columns=fullflexible]
$ sudo reboot
\end{lstlisting}

\item Install Pi-hole using the automated script:
\begin{lstlisting}[language=bash, breaklines=true, breakatwhitespace=true, columns=fullflexible]
$ curl -sSL https://install.pi-hole.net | bash
\end{lstlisting}

Follow the installation prompts:
\begin{itemize}
\item \textbf{Pi-hole Automated Installer:} Select OK
\item \textbf{Open Source Software:} Select OK
\item \textbf{Static IP Needed:} Select Continue
\item \textbf{Choose An Interface:} Select wlan0
\item \textbf{Select Upstream DNS Provider:} Choose Google (or another preferred provider)
\item \textbf{Blocklists:} Select Yes
\item \textbf{Enable Logging:} Select Yes
\item \textbf{Select a privacy mode for FTL:} Choose "Show everything"
\item \textbf{Installation Complete:} Select OK
\end{itemize}

\item Access the Pi-hole web interface:
\begin{lstlisting}[language=bash, breaklines=true, breakatwhitespace=true, columns=fullflexible]
https://<YOUR_IP>/admin
\end{lstlisting}

\item Manually configure Raspberry Pi to use Pi-hole:
\begin{lstlisting}[language=bash, breaklines=true, breakatwhitespace=true, columns=fullflexible]
$ sudo nano /etc/resolv.conf
\end{lstlisting}

Replace the contents with:
\begin{lstlisting}[language=bash, breaklines=true, breakatwhitespace=true, columns=fullflexible]
nameserver <YOUR_IP>
\end{lstlisting}

Lock the file to prevent modifications:
\begin{lstlisting}[language=bash, breaklines=true, breakatwhitespace=true, columns=fullflexible]
$ sudo chattr +i /etc/resolv.conf
\end{lstlisting}
 
(To unlock the file for changes, use: \texttt{\$ sudo chattr -i /etc/resolv.conf})

\item Reboot the Raspberry Pi to apply changes:
\begin{lstlisting}[language=bash, breaklines=true, breakatwhitespace=true, columns=fullflexible]
$ sudo reboot
\end{lstlisting}

\noindent \textbf{Optional Enhancements:}
\begin{itemize}

\item Manually configure a Windows device to use Pi-hole:
\begin{enumerate}
\item Open \textbf{Network Connections}.
\item Right-click on your Wi-Fi connection and select \textbf{Properties}.
\item Select \textbf{Internet Protocol Version 4 (TCP/IPv4)} and click \textbf{Properties}.
\item Choose \textbf{Use the following DNS server addresses} and enter:
\begin{itemize}
\item Preferred DNS server: \texttt{\textless YOUR\_IP\textgreater}
\item Alternate DNS server: Leave blank or use another DNS (e.g., 1.1.1.1)
\end{itemize}
\item Click OK to apply changes.
\end{enumerate}

\end{itemize}

\end{enumerate}