\section{Setting Up Pi-hole for Ad Blocking}

\subsection{Overview}
Video: \href{https://www.youtube.com/watch?v=cE21YjuaB6o}{\textbf{\color{blue}World's Greatest Pi-hole Tutorial - Easy Raspberry Pi Project!}} \\
Channel: \href{https://www.youtube.com/@CrosstalkSolutions}{\textbf{\color{blue}Crosstalk Solutions}} \\
Operating System: Raspberry Pi OS Lite (64-bit)

Pi-hole is a network-wide ad blocker that acts as a DNS sinkhole, filtering out advertisements before they reach your devices. This guide will walk you through setting up Pi-hole on a Raspberry Pi and configuring your network to use it as the primary DNS server.

\subsection{Steps}
\begin{enumerate}
    \item Update your system:
    \begin{lstlisting}[language=bash, breaklines=true, breakatwhitespace=true, columns=fullflexible]
    $ sudo apt update && sudo apt upgrade -y
    \end{lstlisting}

    \item Install the DHCP client daemon (needed for a static IP setup):
    \begin{lstlisting}[language=bash, breaklines=true, breakatwhitespace=true, columns=fullflexible]
    $ sudo apt install dhcpcd
    \end{lstlisting}
    
    \item Configure a static IP address:
    \begin{lstlisting}[language=bash, breaklines=true, breakatwhitespace=true, columns=fullflexible]
    $ sudo nano /etc/dhcpcd.conf
    \end{lstlisting}

    Add the following lines at the end of the file:
    \begin{lstlisting}[language=bash, breaklines=true, breakatwhitespace=true, columns=fullflexible]
    interface wlan0
    static ip_address=<YOUR_IP>/24
    static routers=<YOUR_ROUTER_IP>
    static domain_name_servers=<DNS_SERVER_1> <DNS_SERVER_2>
    \end{lstlisting}

    Save the file and reboot the Raspberry Pi:
    \begin{lstlisting}[language=bash, breaklines=true, breakatwhitespace=true, columns=fullflexible]
    $ sudo reboot
    \end{lstlisting}

    \item Install Pi-hole using the automated script:
    \begin{lstlisting}[language=bash, breaklines=true, breakatwhitespace=true, columns=fullflexible]
    $ curl -sSL https://install.pi-hole.net | bash
    \end{lstlisting}

    Follow the installation prompts:
    \begin{itemize}
        \item \textbf{Pi-hole Automated Installer:} Select OK
        \item \textbf{Open Source Software:} Select OK
        \item \textbf{Static IP Needed:} Select Continue
        \item \textbf{Choose An Interface:} Select wlan0
        \item \textbf{Select Upstream DNS Provider:} Choose Google (or another preferred provider)
        \item \textbf{Blocklists:} Select Yes
        \item \textbf{Enable Logging:} Select Yes
        \item \textbf{Select a privacy mode for FTL:} Choose "Show everything"
        \item \textbf{Installation Complete:} Select OK
    \end{itemize}

    \item Access the Pi-hole web interface:
    \begin{lstlisting}[language=bash, breaklines=true, breakatwhitespace=true, columns=fullflexible]
    https://<YOUR_IP>/admin
    \end{lstlisting}

    \item Configure devices to use Pi-hole manually (Example for Windows):
    \begin{enumerate}
        \item Open \textit{Network Connections}.
        \item Right-click on your Wi-Fi connection and select \textit{Properties}.
        \item Select \textit{Internet Protocol Version 4 (TCP/IPv4)} and click \textit{Properties}.
        \item Choose \textit{Use the following DNS server addresses} and enter:
        \begin{itemize}
            \item Preferred DNS server: \texttt{\textless YOUR\_IP\textgreater}
            \item Alternate DNS server: Leave blank or use another DNS (e.g., 1.1.1.1)
        \end{itemize}
        \item Click OK to apply changes.
    \end{enumerate}

    \item Ensure Raspberry Pi uses Pi-hole for its own DNS queries:
    \begin{lstlisting}[language=bash, breaklines=true, breakatwhitespace=true, columns=fullflexible]
    $ sudo nano /etc/resolv.conf
    \end{lstlisting}

    Replace the contents with:
    \begin{lstlisting}[language=bash, breaklines=true, breakatwhitespace=true, columns=fullflexible]
    nameserver <YOUR_IP>
    \end{lstlisting}

    Lock the file to prevent modifications:
    \begin{lstlisting}[language=bash, breaklines=true, breakatwhitespace=true, columns=fullflexible]
    $ sudo chattr +i /etc/resolv.conf
    \end{lstlisting}
 
    (To unlock the file for changes, use: \texttt{\$ sudo chattr -i /etc/resolv.conf})

    \item Reboot the Raspberry Pi to apply changes:
    \begin{lstlisting}[language=bash, breaklines=true, breakatwhitespace=true, columns=fullflexible]
    $ sudo reboot
    \end{lstlisting}

\end{enumerate}